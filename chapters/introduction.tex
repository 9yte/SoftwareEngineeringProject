\فصل { مقدمه}

\قسمت {شرح مسئله}
در هر سازمانی با گذشت زمان، عواملی وجود دارند که بر طول عمر سیستم‌های موجود در سازمان تاثیر به‌سزایی خواهند گذاشت. مهم‌ترین این عوامل عبارتند از نیازمندی‌های ذی‌نفعان سیستم، سیاست‌های سازمان ، محدودیت‌های تکنولوژی سازمان، تغییرات سریع تکنولوژی و شرایط بازار. ممکن است با گذشت زمان سیاست‌های سازمان به گونه‌ای تغییر کند که دیگر نتوان از سیستم موجود به‌صورت بهینه استفاده کرد و نیاز باشد تغییراتی در آن حاصل شود. هم‌چنین این نیاز می‌تواند به دلیل تغییر در نیازمندی‌های کاربر نیز باشد. یکی از دلیل‌های تغییر نیازمندی‌های ذی‌نفعان می‌تواند ظهور تکنولوژی‌‌های جدید باشد که سیستم قدیمی می‌بایستی با این تکنولوژی‌ها وفق پیدا کند. هم‌چنین امکان دارد سازمان به هر دلیلی با محدودیت‌های تکنولوژی مواجه شود که منجر به ناکارآمد بودن سیستم قدیمی باشد.نکته قابل توجه در این موارد اینجاست که قدیمی شدن یک سیستم به اندازه‌ی طول عمر آن فقط وابسته نیست، بلکه امکان دارد یک سیستم که ۶ ماه است شروع به کار کرده است به دلیل تصمیمات راهبردی در سطوح بالای سازمان برای اتخاذ تکنولوژی جدید یا ایجاد نیازمندی‌های جدید به عنوان یک سیستم قدیمی شناخته شود. سیستم قدیمی سیستمی است که دیگر پاسخگوی نیازمندی‌های موردنظر سازمان نیست یا نمی‌تواند نتیجه‌ی مدنظر را برای سازمان و مشتری به ارمغان بیاورد. به همین دلایل نیاز به ایجاد تغییرات در سازمان و انتقال سیستم قدیمی مد نظر به سیستم جدید ایجاد می‌شود.
\\
عمدتا سیستم‌های قدیمی در یک سازمان به دلایل منطقی متعدد نگهداری می‌شوند و این دلایل عبارتند از:
\شروع{فقرات}
\فقره سیستم به خوبی کار می‌کند و دلیلی برای تعویض یا  توسعه‌ی آن دیده نمی‌شود.
\فقره هزینه‌ی طراحی یا جایگزینی سیستم دیگر زیاد است.
\فقره سیستم باید همواره در دسترس باشد مانند سیستم‌های موجود در بانک‌ها و مراقبت پرواز.
\فقره نحوه‌ی کار سیستم به خوبی درک نمی‌شود، زیرا سیستم به خوبی مستندسازی نشده است. 
\پایان{فقرات}
با این وجود سیستم‌های قدیمی علاوه بر هزینه‌ی نگهداری و مناسب نبودن با راهبردهای سازمان، مشکل‌های دیگری نیز ایجاد می‌کنند مانند:
سخت شدن کار با سیستم و به دلیل اموزش ندیدن افراد برای کار با این 
سیستم‌ها و یکپارچه‌سازی دشوار سیستم با سایر سیستم‌های موجود در سیستم به دلیل استفاده از یک  تکنولوژی خاص.
با توجه به این موارد انتقال سیستم‌های قدیمی در شرایط مختلف به یک صورت نخواهد بود بنابراین نیاز است که برای هر سیستم و در شرایط مختلف روش‌های  مناسب با آن موقعیت مورد استفاده قرار گرفته شود. در این پژوهش مسئله‌ی انتقال سیستم‌های قدیمی به سیستم‌های جدید مد نظر قرار دارد که با توجه به هر وضعیت بر مبنای یک رویکرد کلی، سبدی از روشها به دست آید تا با استفاده از آن بتوان بهترین روش‌ها در آن زمینه رو اتخاذ کرد.
\قسمت{اهمیت}
در دنیای امروز موفقیت یک پروژه وابسته به هزینه و زمان صرف شده برای اجرای آن و کیفیت نهایی پروژه در دیدگاه مشتری است. اهمیت تشخیص راه حل مناسب برای انجام یک فعالیت در میزان هزینه و زمان اجرای آن هویدا می‌گردد چرا که هر قدر که پروژه پیشرفت کند انجام تغییرات در آن هزینه‌ی بیش‌تری خواهد داشت و باعث به درازا کشیده شدن آن می‌گردد و مشاهده چند بار انجام دادن یک پروژه است. این مسائل علاوه بر تاثیر روی بودجه‌ی یک پروژه بر روی انگیزه‌ی افراد نیز موثر خواهد بود.
\\
تغییرهای بنیادین در سیستم‌های اصلی یک سازمان همواره کاری پیچیده تلقی می‌شده است و در برابر آن انواع مقاوت‌ها وجود داشته است. مهم‌ترین دلیل این مقاومت‌ها جلوگیری از تغییر در یک سیستم پاسخگوی نیازها است. این سیستم‌ها کارایی قابل قبولی ارائه می‌دهند و در عین حال کار با آن برای کارکنان سازمان مشخص است. نکته‌ی حائز اهمیت این است که همیشه برای انتقال سیستم‌های قدیمی، روش جایگزینی آن سیستم با یک سیستم جدید مطرح نیست و راهکارهای دیگری با توجه به نوع سیستم، سازمان و شرایط موجود و مطلوب نیز موجود است، اما عملا به دلیل وجود نداشتن مختصص در یک سازمان و درنتیجه دانش کم در این حوزه  آن راهکارها در نظر  گرفته
نمی‌شود و انتقال سیستم‌های قدیمی به عنوان یک معضل بزرگ در سازمان باقی می‌ماند. هزینه‌ی استخدام متخصص برای سازمان می‌تواند به صرفه نباشد یا هزینه‌ی کل یک پروژه را افزایش دهد. بنابراین با استفاده از یک مستند مناسب و یک مدل ارزیابی قابل فهم، افراد مختلف با دانش کم در این حوزه نیز می‌توانند تصمیم‌گیری‌های خوب و دقیقی را انجام دهند.
\\
 با طبقه‌بندی مناسب انواع روش‌ها می‌توان نیاز به دانش در یک سازمان کم‌تر شود، زیرا یک مستند مناسب شامل تمام روش‌ها در اختیار قرار دارد که مانند یک مرجع عمل می‌کند و در شرایط مختلف افراد با مراجعه‌ی به آن می‌توانند گزینه‌ی مناسب با  شرایط خود را پیدا کنند. این کار با تعریف یک سری شاخص و  یک مدل ارزیابی انجام می‌شود که در فصل‌های آتی به تفضیل توضیح داده می‌شود.
\قسمت{انگیزه}
انگیزه‌ی اصلی برای  انجام این  پروژه به دو بخش کلی تقسیم می‌شود:
\\
تهیه یک کتابچه‌ی راهنما برای انواع روش‌ها و تکنیک‌های موجود انتقال سیستم‌های قدیمی که در پژوهش‌ها و کتب معرفی شده‌اند، که در قالب  این راهنما موارد مورد نیاز برای استفاده از هر روش و تکنیک با توجه به موارد مختلف توضیح داده می‌شود، این موارد عبارتند از  معماری مورد استفاده در سیستم مبدا یا سیستم مقصد، شرایط سازمانی ، ویژگی‌های کارکردی سیستم، طول عمر و سایر ویژگی‌ها که در فصل دوم و سوم مفصل توضیح داده می‌شود. انگیزه‌ی اصلی در این پژوهش استانداردسازی تعاریف موجود در حوزه‌ی انتقال سیستم‌های قدیمی است که در طی آن انواع روش‌ها و تکنیک‌ها به صورت مجزا بر اساس نوع تغییرات مورد استفاده در ساختار یا پیکربندی تعریف می‌گردند. در این تعاریف برای بهبود امکان ارجاع به آن از یک ساختار مشخص استفاده می‌شود تا همه‌ی افراد با سطح دانش متفاوت در حوزه‌ی نرم افزار بتوانند به راحتی از آن استفاده کنند. همچنین انگیزه‌ی دیگر، جمع آوری تمامی روش‌ها و تکنیک‌های کار شده در پژوهش‌های مختلف می‌باشد تا  در نهایت یک طبقه‌بندی کلی براساس رویکردهای اصلی ایجاد شود و این روش‌ها در قالب این رویکردها معرفی گردند.
\\
انگیزه‌ی دوم معرفی یک راه حل برای انتخاب مناسب‌ترین تکنیک از میان روش‌های ارائه شده است  تا با توجه به وضعیت سیستم  و سازمان بهترین روش انتقال اتخاذ گردد. همان‌طور که گفته شد عموما برای انتقال یک سیستم قدیمی روش جایگزینی یا ساخت مجدد سیستم از طرف افراد انتخاب می‌شود که این روش هزینه‌ی بسیار بالایی می‌تواند داشته باشد،‌ این هزینه‌ها شامل هزینه‌ی اجرای یک پروژه‌ی کامل تولید یک سیستم جدید، آموزش منایع انسانی برای کار با سیستم جدید و وجود نداشتن سیستم در یک بازه‌ی زمانی تا اماده شدن سیستم جدید می‌باشد. روش‌های دیگر انتقال سیستم‌ها که در سه رویکرد کلی جای می‌گیرند دارای مزایا و معایب جداگانه‌ای هستند که در این میان می‌توان بهترین روش برای هر موقعیت را بر اساس شاخص‌های تعریف شده، یافت. به صورت خلاصه برای این کار باید تعدادی شاخص تعریف شود تا با توجه به آن شاخص‌ها بتوان یک ارزیابی مناسب انجام داد و یک مدل ارزیابی به دست آورد که در وهله‌ی  بعدی با این مدل روش مناسب را انتخاب کرد. 

\قسمت{سابقه پژوهش}
ارجاع به مقالات که هنوز کامل نیست
\قسمت{هدف پژوهش}
همانطور که در دو قسمت اهمیت و انگیزه گفته شد جمع آوری روش‌های انتقال سیستم‌های قدیمی و طبقه آن هدف اصلی پژوهش است. برای نیل به این هدف می‌بایست در وهله‌ی اول نیازمندی‌های انتقال یک سیستم قدیمی، ریسک‌های این انتقال،  رویکردهای انتقال و شرایط ایجاد کننده این انتقال شناسایی شوند که هر یک از این موارد به صورت جداگانه  به عنوان یک هدف برای رسیدن به هدف اصلی پژوهش در نظر گرفته می‌شود. جمع‌آوری و طبقه‌بندی این موارد همگی در قالب یک کتابچه راهنما به دست خواهد آمد که مرجعی برای انتقال سیستم‌های قدیمی در یک سازمان ‌می‌باشد تا نیاز به دانش در یک سازمان کم‌تر شود و احتمال موفقیت انتقال سیستم‌ها از جوانب مختلف هزینه، زمان و 
مقاومت سازمانی افزایش یابد و ریسک‌های همراه با این انتقال پاسخ مناسبی خواهند داشت.
\\
در بخش دوم پژوهش هدف پیدا کردن شاخص‌های مناسب برای تعریف یک مدل ارزیابی روش‌ها می‌باشد، که این شاخص‌ها براساس شاخص‌های موجود در پژوهش‌ها و طرح پرسشنامه از متخصصین این حوزه به دست می‌آید. با معرفی شاخص‌ها هدف طراحی یک مدل ارزیابی خواهد بود، که در طی آن  بتوان شرایط موجود در سیستم و سازمان را ارزیابی کرد و سپس با ارزیابی به دست آمده و مقایسه با حالت‌های ایده آل برای هر روی‌کرد می‌توان روی‌کرد مناسب برای یک انتقال را به دست آورد و بر اساس آن رویکرد سبدی از روش‌ها در اختیار تیم انتقال خواهد بود که با استفاده از آن، روش انتقال را با مناسب‌ترین روش انجام دهند.
\قسمت{روش پژوهش}
با توجه به این که در بخش اول پژوهش عمده‌ی کار جمع‌آوری روش‌های موجود است، بنابراین بهترین روش برای انجام این بخش مطالعه‌ی مقاله‌های معتبر و کتب مرجع این حوزه است که در بخش منابع این رساله، عمده مقاله‌ها معتبر در این حوزه آورده شده است. در این روش مقاله‌های مختلف در حوزه‌های گوناگون با استفاده از کلمه‌های کلیدی و عنوان  جست و جو کرده و سپس از میان   آنها مقاله‌های مرتبط را مطالعه کرده و در طبقه‌بندی قرار دادیم. مرحله‌ی دوم با انتخاب شاخص‌های موجود در مقاله‌ها برای تعریف مدل ارزیابی پی گرفته می‌شود و در اینجا برای سایر شاخص‌ها از افراد متخصص کمک گرفته می‌شود که این کار با طراحی یک پرسش‌نامه و پر کردن آن به وسیله‌ی متخصصین پی گرفته می‌شود و با استفاده از یک سری متد که در فصل  چهارم توضیح داده خواهد شد این شاخص‌ها ارزیابی می‌شوند.
\قسمت{ساختار}
این رساله از ۵ فصل تشکیل شده‌است که در فصل اول آن مقدمه و موضوع پژوهش توضیح داده شده است. در این فصل  انگیزه و هدف  برای انجام این پژوهش  آورده شده‌است. حوزه‌ی مسئله و شرح دقیق مشکل و شرایط ایجاد کننده‌ی آن به صورت مشروح نیز در ابتدای این فصل قرار دارد.
\\
در فصل دوم هدف ارائه‌ی تعاریف مفاهیم اصلی مورداستفاده در این پژوهش است تا با استفاده از یک تعریف استاندارد و مشترک برای نویسندگان و خوانندگان،  برقراری ارتباط با بخش‌های بعدی  را ساد‌ه‌تر کند. در این فصل مفاهیم کلی و طبقه‌بندیهای اولیه توضیح داده می‌شود.
\\ 
در فصل سوم کارهای پیشین انجام شده در این حوزه مورد بررسی قرار می‌گیرد و در این بخش  منابع مختلف که در این حوزه مطالعاتی انجام داده‌اند، آورده می‌شود. این منابع شامل رساله‌های دانشگاهی در مقاطع مختلف، مقاله‌های معتبر و کتب مرجع هستند و پس از  بررسی این کارها در مرحله‌ی بعدی این مطالعه‌ها براساس خصوصیت‌های مشترک طبقه‌بندی می‌شوند.
\\
در فصل چهارم شاخص‌ها معرفی می‌شوند. در ابتدای این بخش شاخص های مختلف که روش جست و جو یا پرسش‌نامه از متخصصین به دست آمده است معرفی می‌گردد و در ادامه مدل ارزیابی طراحی شده با این شاخص‌ها شرح داده می‌شود. در این قسمت تعدادی مثال برای مشخص شدن نحوه‌ی استفاده از این مدل ارزیابی آورده می‌شود.
\\
فصل پنجم و آخر نتیجه گیری را در بر دارد که در این فصل نتایج کلی از پژوهش توضیح داده می‌شود و کارهای آتی ممکن با استفاده از این پژوهش نیز مورد بررسی قرار می‌گیرد. 