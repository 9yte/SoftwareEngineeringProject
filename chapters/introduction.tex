\فصل { مقدمه}

\قسمت {شرح مسئله}
در هر سازمانی با گذشت زمان، پارامترهایی وجود دارند که بر طول عمر سیستم‌های موجود در سازمان تاثیر به‌سزایی خواهد گذاشت. مهم‌ترین این پارامترها عبارتند از نیازمندی‌های ذی‌نفعان سیستم، سیاست‌های سازمان و محدودیت‌های تکنولوژی سازمان. ممکن است با گذشت زمان سیاست‌های سازمان به گونه‌ای تغییر کند که دیگر نتوان از سیستم موجود به‌صورت بهینه استفاده کرد و نیاز باشد تغییراتی در آن حاصل شود. هم‌چنین این نیاز می‌تواند به دلیل تغییر در نیازمندی‌های کاربر نیز باشد. یکی از دلیل‌های تغییر نیازمندی‌های ذی‌نفعان می‌تواند ظهور تکنولوژی‌‌های جدید باشد که سیستم قدیمی می‌بایستی با این تکنولوژی‌ها وفق پیدا کند. هم‌چنین ممکن است سازمان به هر دلیلی با محدودیت‌های تکنولوژی مواجه شود که منجر به ناکارآمد بودن سیستم قدیمی باشد. به‌طور کلی می‌بایستی بتوان برای مواجه با این مشکل‌ها، راه‌حلی با کم‌ترین هزینه اتخاذ گردد تا سیستم جدید به خوبی پاسخگوی نیازمندی‌های ذی‌نفعان پروژه باشد.
\قسمت{اهمیت}

\قسمت{انگیزه}

\قسمت{سابقه پژوهش}

\قسمت{هدف پژوهش}

\قسمت{روش پژوهش}


\قسمت{ساختار}
