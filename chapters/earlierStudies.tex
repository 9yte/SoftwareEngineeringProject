\فصل { کارهای پیشین}
برای انتقال سیستم‌های موروٍثی به سیستم‌هام مدرن به‌طور کلی ۴ روی‌کرد جایگزینی\پانویس{replacement}، مهندسی مجدد\پانویس{reengineering}، مهاجرت\پانویس{migration} و پوشاندن\پانویس{wrapping} تقسیم کرد.
\قسمت{جایگزینی}
در این رویکرد سیستم موروثی توسط سیستم جدیدی جایگزین می‌شود و از هیچ‌یک از اجزای ان استفاده‌ای نمی‌شود. به عبارتی سیستم دوباره نوشته می‌شود. اگرچه که این روش توصیه نمی‌شود، اما گاهی ۳ روش دیگر هزینه‌هایی را متحمل می‌شوند که قابل‌ تخمین نمی‌باشند. نوشتن دوباره‌ی برنامه اگرچه امری‌ پرهزینه، مخاطره‌آمیز و زمان‌گیر می‌باشد، اما در نهایت می‌تواند منجر به سیستمی شود که دقیقا نیازهای سازمان را برطرف می‌کند.
 
سیستم‌های تجاری خارج از قفسه\پانویس{Commercial off-the-shelf (COTS)} محصول‌های نرم‌افزاری آماده‌ای می‌باشند که در بازار موجود می‌باشند. یکی از روش‌های جایگزینی این می‌باشد که برنامه‌ي موجود را با یک مولفه‌ی COTS جایگزین کنیم. اگرچه این روش در زمان صرفه‌جویی می‌کند و خطر کم‌تری دارد،‌ اما به دلیل تغییراتی که در آینده اجتناب‌ناپذیر می‌باشد ممکن است هزینه‌ی سنگینی را بر سازمان متحمل کند. هم‌چنین گاهی کسب و کار مهمی در سیستم‌ موجود، نهفته می‌باشد و استفاده از این مولفه‌ها امری هزینه‌بر می‌باشد، زیرا ممکن است نیاز به اعمال تغییراتی در مولفه‌ها باشد. \مرجع{almonaies2010legacy}

این روش به گفته‌ی Comella-Dorda دو خطر مهم را دارا می‌باشد:
\شروع{فقرات}
\فقره نگه‌داری سیستم جدید که به‌خوبی سیستم قبل شناخته‌شده نمی‌باشد.
\فقره نبود یک ضمانت که سیستم جدید عملکرد سیستم قبلی را دارد یا خیر. \مرجع{comella2000survey}
\پایان{فقرات}
\قسمت{مهنسدی مجدد}
مهندسی مجدد عبارت است از تحلیل و اصلاح یک برنامه به منظور نمایش آن در یک فرم جدید. این روش می‌تواند شامل فعالیت‌هایی از قبیل مهندسی معکوس\پانویس{Reverse Engineering}، بازسازی\پانویس{Restructuring}، طراحی مجدد\پانویس{Redesigning} و پیاده‌سازی مجدد\پانویس{Re-implementing} باشد.

همواره سه مسئله‌ی اساسی در مهندسی مجدد سرویس‌گرا وجود دارد که عبارتند از شناسایی سرویس\پانویس{Service identification}، بسته‌بندی سرویس \پانویس{Service packaging} و استقرار سرویس\پانویس{Service deployment}. مهندسی مجدد سرویس‌گرا برای سیستم‌های موروثی با خصوصیات زیر  مناسب می‌باشد:
\شروع{فقرات}
\فقره برخی از مولفه‌های سیستم موروثی بیش‌تر از کل سیستم قابلیت نگه‌داری دارند.
\فقره عملکرد نفهته در سیستم به عنوان یک سرویس مجزا می‌تواند مفید واقع شود.
\فقره برخی از مولفه‌های سیستم موجود می‌توانند به تدریج جایگزین شوند بدون تاثیر منفی‌ای بر روی مصرف‌کننده‌ی سرویس.
\فقره سیستم موروثی نیاز دارد که به یک محیط توزیع‌شده مهاجرت کند.
\فقره سیستم موجود عملکرد قابل اطمینان و با قابلیت استفاده‌ی مجدد دارد که منطق کسب‌ و کار با ارزشی دارد.
\فقره مولفه‌های موجود نیاز دارند که روی زیرساخت\پانویس{Platform}‌های مختلفی اجرا شوند.\مرجع{almonaies2010legacy}
\پایان{فقرات}
\زیرقسمت{نگاهی بر برخی از شیوه‌های مهندسی مجدد}
Chung  و سایرین پروژه‌ای را تعریف کردند که در آن یک ابزار موروثی استنتاج\پانویس{Derivation} و بررسی اثبات فرضیه دوباره مهندسی شده‌است و یک سیستم سرویس‌گرا به‌دست آمده‌است. این ابزار جدید که SoBertie نام دارد، توانایی‌های مرکزی ابزار اصلی را به عنوان یک سرویس تحت وب دارا می‌باشد. \مرجع{chung2005service}

هم‌چنین او و سایرین یک فراروش\پانویس{Methodology} سرویس‌گرا مهندسی مجدد، به‌منظور اعمال معماری سرویس‌گرا به سیستم‌های موروثی تعریف کرده‌اند. این فراروش معماری محور\پانویس{Architecture-centric}، سرویس‌گرا، مختص به نقش\پانویس{Role-specific} و مدل‌رانه\پانویس{Model-driven} می‌باشد. اگرچه مثال کاملی که از این فراروش استفاده کند هنوز وجود ندارد، اما یک بررسی موردی\پانویس{Case study} از یک سیستم موجودی یک مغازه‌ی خرده‌فروشی بیان شده‌است.\مرجع{chung2007service}

یک رویکرد جامع توسط Distante و سایرین به منظور طراحی مجدد برنامه‌های موروثی برای وب ارائه شده‌است. در این رویکرد از چارچوب‌های UWA\پانویس{Ubiquitous Web Applications Design Framework} و UWAT+ (نسخه‌ی توسعه‌یافته‌ی مدل طراحی تراکنش\پانویس{Transaction Design Model})استفاده می‌شود. آن شامل تکنولوژی‌های بازیابی طراحی برای برنامه‌های قدیمی می‌باشد و روش‌هایی طراحی برای سیستم‌های تحت وب فراهم می‌کند. به‌طور کلی این فرآیند شامل سه‌مرحله‌ی زیر می‌باشد:
\شروع{فقرات}
\فقره استخراج نیازمندی‌ها\پانویس{Requirements elicitation}
\فقره مهندسی معکوس
\فقره مهندسی رو به جلو\پانویس{Forward engineering}\مرجع{distante2006towards}
\پایان{فقرات}

Chen و سایرین با استفاده از تحلیل ویژگی به مهندسی مجدد سرویس‌گرا می‌پردازد. تحلیل ویژگی شامل شناخت ویژگی‌های سیستم، ساخت یک مدل ویژگی به منظور سازمان‌دهی ویژگی‌ها و شناسایی پیاده‌سازی آن‌ها در سیستم موروثی. آن‌ها از یک سیستم مدیریت اطلاعات\پانویس{Management information system(MIS)} به عنوان بررسی موردی استفاده کردند. MIS با استفاده از یک تکنیک بالا به پایین\پانویس{Top-down} تجزیه‌ی دامنه و تحلیل ویژگی، مورد بررسی قرار می‌گیرد. با استفاده از این ویژگی‌ها سرویس‌ها شناسایی می‌شوند و سپس توسط یک ابزار این سرویس‌ها پیاده‌سازی می‌شوند. این ابزار توانایی تولید کد چسبناک\پانویس{Glue code} برای سرویس‌های وب و کدمنبع‌های مرتبط به روش‌های سرویس وب دارد.\مرجع{chen2005feature}

Cuadrado و سایرین یک فرآیند به منظور بهبود\پانویس{Recovery} معماری سیستم‌های موروثی به هدف شناخت طرح مورد استفاده در نوین کردن سیستم موجود. به‌طور کلی از یک رویکرد جعبه‌-سفید\پانویس{White-box} برای تغییر کد موروثی موجود استفاده شده‌است. این رویکرد شامل سه بخش بهبود معماری موروثی، ساخت طرح تکامل و اجرای طرح می‌باشد. بهبود معماری به ساخت یک مستند مناسب کمک می‌کند. طرح تکامل شامل ۴ فاز انتخاب معماری، تعریف چرخه‌های تکامل، برنامه‌ریزی چرخه‌ها و یک امکان‌سنجی مقدماتی. \مرجع{cuadrado2008case}
\قسمت{پوشاندن}