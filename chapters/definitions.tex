\فصل { ادبیات موضوعی و تعاریف}

\قسمت {سیستم‌های موروثی}
در حال حاضر به‌طور کلی تعریف یکتا و مشخصی از این سیستم‌ها وجود ندارد. ما در این رساله سیستم‌های موروثی را سیستم‌هایی توصیف می‌کنیم که برای یک بنگاه مهم می‌باشد و در عین حال بنگاه نمی‌داند که چگونه آن را به‌روزرسانی کند تا نیازهای جدید را برآورده کند. \مرجع{chen2013legacy}
\زیرقسمت{ریسک‌های استفاده از سیستم‌های موروثی}
در استفاده از سیستم‌های موروثی ریسک‌هایی وجود دارد. در ادامه به توضیح برخی از مهم‌ترین این ریسک‌ها می‌پردازیم.
\شروع{فقرات}
\فقره هزینه‌ی نگه‌داری این سیستم‌ها بالا می‌باشد. علت اصلی این هزینه‌ی بالا ساختار ضعیف‌شده‌ی این سیستم‌ها به دلیل نگه‌داری‌های بلندمدت می‌باشد.
\فقره پیاده‌سازی تغییرات نیازمند درک بالایی از ساختار سیستم می‌باشد. سیستم‌های موروٍثی دارای ساختار پیچیده‌ای می‌باشند و فهم آن‌ها تلاش زیادی را متحمل می‌شود.
\فقره این سیستم‌ها شامل اجزای سخت‌افزاری و نرم‌افزاری منسوخی می‌باشند، اجزایی که تصحیح آن‌ها امری ناممکن می‌باشد. برای جلوگیری از این ریسک می‌توان در هنگام به‌هنگام‌سازی اجزای سیستم، از اجزایی استفاده کرد که نه‌تنها امروز، بلکه در دوره‌ی حیات سیستم مورد حمایت قرار گیرد.
\فقره بر اساس دوره‌ی حیات سیستم، افراد مختلفی آن را نگه‌داری خواهند کرد. پرسنل باتجربه‌ی نگه‌داری یکی از ارزنده‌ترین دارایی‌ها می‌باشند، اما متاسفانه همیشه در دسترس نخواهند بود. به همین دلیل ممکن است در مقاطعی از پرسنل کم‌تجربه‌ استفاده شود و این مهم می‌تواند هزینه‌ی نگه‌داری را بالا ببرد، زیرا فهم سیستم امری زمان‌گیر خواهد بود.
\فقره این سیستم‌ها معمولا دارای رویه‌ها، قانون‌ها و دانش کسب‌ و کار می‌باشد که ممکن‌است به‌صورت مشهودی مستندسازی نشده باشند. پس در هنگام به‌هنگام سازی به‌خصوص در روش‌های جایگزینی ممکن‌است برخی از این اطلاعات مهم از بین بروند که خطر بزرگی برای آینده‌ی سیستم خواهند بود.
\فقره خیلی از این سیستم‌ها مستندهای ضعیفی دارند، زیرا این مستند‌ها به خوبی به‌هنگام نشده‌اند. به‌خصوص تصحیح عیب‌های کوچک سیستم معمولا مستند نمی‌شوند. حتی در برخی موارد مستند‌های سیستم دچار ناسازگاری می‌شوند. در مواردی حتی دیده‌می‌شود که نقص‌هایی در کدمنبع سیستم پدید می‌آید.  \مرجع{warren2012renaissance}
\پایان{فقرات}

\قسمت{سیستم‌های جدید}
همان‌طور که گفته‌شد بر اساس شرایط و نیازهای جدید گاهی واجب می‌شود که سیستم‌های موروثی تکامل پیدا کنند و به سیستم‌هایی که جدید نامیده می‌شوند تبدیل شوند.

\قسمت{سسیستم‌های تحت وب و موبایل}
امروزه از WWW برای اجرای برنامه‌های با ابعاد بزرگ به منظور فعالیت‌های مختلفی اعم از تجارت، توزیع اطلاعات و کارهای گروهی استفاده می‌شود. برنامه‌های وب بر روی پلت‌فرم‌‌های سخت‌افزاری اجرا می‌شوند که توزیع شده می‌باشند. هم‌چنین این برنامه‌ها توسط نرم‌افزار‌هایی حمایت می‌شوند که:
\شروع{فقرات}
\فقره توزیع‌شده می‌باشند
\فقره در چند زبان پیاده‌سازی شده‌اند
\فقره با کاربران، وب‌گاه‌های دیگر و پایگاه‌داده‌ها تعامل دارند
\فقره شامل اجزاهایی می‌باشد که قابلیت استفاده‌ی مجدد را دارند
\پایان{فقرات}
رشد چشم‌گیر در استفاده از این برنامه‌ها، این‌گونه برنامه‌ها را تبدیل به یکی از مهم‌ترین‌ و بزرگ‌ترین قسمت‌های صنعت نرم‌افزار کرده‌است. این سیستم‌ها از اجزای مجزایی ساخته‌می‌شوند که این اجزا از منبع‌های متفاوتی مشتق می‌شوند.  \مرجع{offutt2002quality}
\\
حال به تعریف سیستم‌های تحت موبایل می‌پردازیم.
\زیرقسمت{سیستم‌های دارای معماری سرویس‌گرا}
در این معماری ارتباط‌های انعطاف‌پذیری بین اجزای مختلف سیستم وجود دارد تا با تغییرات در کسب و کار بتوان مقابله کرد. این معماری با جداسازی رابط از پیاده‌سازی داخلی، بر روی تبادل اطلاعات میان اجزای اصلی نرم‌افزار و قابلیت استفاده‌ی مجدد از اجزا تمرکز می‌کند. در ادامه به برخی از ویژگی‌های این معماری می‌پردازیم:
\شروع{فقرات}
\فقره انعطاف‌پذیری بالا
\فقره خودمختاری \پانویس {autonomy}
\فقره قابلیت استفاده مجدد
\فقره بدون حالت \پانویس {statelessness}
\فقره چابکی  \پانویس {agility}
\فقره ؟؟؟؟؟؟؟؟؟\مرجع{almonaies2010legacy}
\پایان{فقرات}
به‌طور کلی این معماری از لایه‌های زیر تشکیل شده‌است:
\شروع{فقرات}
\فقره لایه‌ی سیستم‌های عملیاتی \پانویس {Operational Systems Layer}: این لایه شامل برنامه‌های موجود ساخته‌شده اعم از برنامه‌های موروثی، پیاده‌سازی‌های شی‌گرا قدیمی‌تر.
\فقره لایه‌ی اجزای تجاری \پانویس {Enterprise Components Layer}: این لایه به فهم عملکرد و نگه‌داری از کیفیت سرویس \پانویس{Quality of Service} کمک می‌کند. این لایه از تکنولوژی‌های مبتنی بر ظرف استفاده می‌کند تا اجزای موردنظر پیاده‌سازی شوند.
\فقره لایه‌ی سرویس‌ها: این لایه شامل سرویس‌هایی می‌باشد که قرار است فراهم شوند. از این سرویس‌ها می‌توان برای ساخت اجزای تجاری، کسب و کار و حتی پروژه محور استفاده کرد و این اجزا در زمان اجرا با استفاده از رابط‌هایشان عمل می‌کنند.
\فقره لایه‌ی فرآیند کسب و کار \پانویس {Business Process Layer}: ترکیب و طراحی سرویس‌های لایه‌ی سوم، در این لایه مشخص می‌شوند. سرویس‌ها بر اساس نیاز مربوطه، در یک توالی مشخص یک برنامه را تشکیل می‌ٔهند. این برنامه‌ها وظیفه‌مندی‌ها و فرآیندهای کسب و کار را حمایت می‌کنند.
\فقره لایه‌ی نمایش  \پانویس {Presentation Layer}: این لایه شامل رابط‌های مختلفی برای کاربران یا برنامه‌ها به هدف دسترسی به سرویس‌ها می‌باشد.
\فقره لایه‌ی تجمیع \پانویس {Integration Layer}: این لایه براساس توانایی سرویس‌ها، تجمیع آن‌ها را پشتیبانی می‌کند.
\فقره لایه‌ی کیفیت سرویس: این لایه به رصد، مدیریت و نگه‌داری از کیفیت سرویس اعم از امنیت کمک می‌کند. \مرجع{arsanjani2004service}
\پایان{فقرات}
\زیرقسمت{سیستم‌های دارای معماری عامل‌گرا}

\زیرقسمت{سایر معماری‌ها}

\قسمت{تکنیک‌های انتقال سیستم‌ها}

\زیرقسمت{بازسازی }

\زیرقسمت{پیکربندی مجدد}

\زیرقسمت{مهندسی مجدد}

